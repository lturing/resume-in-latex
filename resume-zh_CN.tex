% !TEX TS-program = xelatex
% !TEX encoding = UTF-8 Unicode
% !Mode:: "TeX:UTF-8"
\documentclass{resume}
%\documentclass[8pt]{article}
%\usepackage{zh_CN-Adobefonts_external} % Simplified Chinese Support using external fonts (./fonts/zh_CN-Adobe/)
\usepackage{zh_CN-Adobefonts_internal} % Simplified Chinese Support using system fonts
\usepackage{linespacing_fix} % disable extra space before next section
\usepackage{cite}
\usepackage[colorlinks,linkcolor=blue,anchorcolor=blue,citecolor=green,urlcolor=blue]{hyperref}


%for header image
\usepackage{graphicx}

%for floating figures
\usepackage{wrapfig}
\usepackage{float}
%\floatstyle{boxed}
%\restylefloat{figure}


\newcommand{\vcenteredinclude}[1]{\begingroup
\setbox0=\hbox{\hspace*{-2.9cm}\includegraphics[width=1.2cm]{#1}}%
\parbox{\wd0}{\box0}\endgroup}


\begin{document}
\pagenumbering{gobble} % suppress displaying page number

\name{ 佩\hspace{0.1cm}奇 }

% {E-mail}{mobilephone}{homepage}
% be careful of _ in emaill address
{\centerline{\contactInfo{\href{mailto:peiqi@foxmail.com}{peiqi@foxmail.com}}{(+86)180-8888-8888}{7岁}\vcenteredinclude{images/photo.jpg}}}
%\centerline{\includegraphics[width=0.2cm]{mazhen.jpg}}

% {E-mail}{mobilephone}
% keep the last empty braces!
%\contactInfo{xxx@yuanbin.me}{(+86) 131-221-87xxx}{}



\section{\faGraduationCap 教育背景}

\datedsubsection{\textbf{xx大学}, xx学院}{20xx年9月 -- 20xx年6月}
\textit{硕士}\ \ xx专业

\datedsubsection{\textbf{xx大学}, xx学院}{20xx年9月 -- 20xx年7月}
\textit{学士}\ \ xx专业


\section{\faCogs\ 技能}
% increase linespacing [parsep=0.5ex]
\begin{itemize}[parsep=0.5ex]
  \item 英语等级:CET-6
  \item 熟悉Python、了解C、C++、Java、JavaScript
  \item 对计算机视觉(分类、检测和分割)有一定认识
  \item 对自然语言处理有一定了解
  \item 对常见的数据结构与算法、机器学习算法有一定了解
  \item 对网络爬虫有一定的了解
\end{itemize}


\section{\faUsers\ 科研 / 项目经历}

\datedsubsection{\textbf{利用Yolov3实现手机摄像头目标识别}}{20xx年x月 -- 20xx年x月}
\begin{itemize}
\item 页面上调用摄像头,并利用Ajax将摄像头拍摄的图片动态地发给Tornado后端
\item Tornado后端接收图片后,调用训练好的YoloV3模型来检测图片中的物体并显示
\end{itemize}

\datedsubsection{\textbf{实验室博客平台搭建}}{20xx年x月 -- 20xx年x月}
\begin{itemize}
\item Django作为后端,mysql作为数据库
\item 博文支持markdown
\item 利用爬虫技术,在博客上一键重启实验室wifi路由器、一键开启实验室门锁
\end{itemize}


\section{\faPaperPlane\ 自我评价}
\hspace*{0.8cm}自我评价。。。



%% Reference
%\newpage
%\bibliographystyle{IEEETran}
%\bibliography{mycite}
\end{document}
